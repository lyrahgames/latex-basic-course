\documentclass[fleqn,a4paper]{article}
\usepackage[utf8]{inputenc}
\usepackage[left=20mm,right=20mm,top=20mm,bottom=20mm]{geometry}
\usepackage{amsmath}
\usepackage{mathtools}

\title{Exercise Sheet}
\author{Markus Pawellek}

\newcommand{\dotProduct}[2]{\left\langle #1, #2 \right\rangle}
\newcommand{\diff}[1]{\mathrm{d}#1}
\newcommand{\integral}[4]{\int_{#1}^{#2} #3 \, \diff{#4}}

\begin{document}
  \section{Computations...}
    \begin{align*}
      \dotProduct{h}{V_f^* g}
      &= \dotProduct{V_f h}{g} \\
      &= \integral{G}{}{ \integral{G}{}{h(x) \overline{f(y^{-1}x)}}{x}\ \overline{g(y)} }{y} \\
      &= \integral{G}{}{ h(x) \overline{\integral{G}{}{g(y)f(y^{-1}x)}{x}}\ }{y}
    \end{align*}

  \section{Mechanics}
    The symmetry character of a state does not change in the course of time:
    \[
      \psi(t) = T e^{-\frac{i}{\hbar}\integral{0}{t}{H(t')}{t'}} \psi(0)
    \]
    For arbitrary permutations $P$, the states introduced in the last section satisfy
    \begin{align*}
      P\psi_s &= \psi_s \\
      P\psi_a &= (-1)^P \psi_a
    \end{align*}
    with
    \[
      (-1)^P =
      \begin{cases}
        1 &: \text{for even permutations} \\
        -1 &: \text{for odd permutations}
      \end{cases}
    \]
    Thus, the state $\psi_s$ and $\psi_a$ form the basis of two one-dimensional representations of the permutation group $S_N$.
    For $\psi_s$, every $P$ is assigned the number $1$, ...

  \subsection{Gauss sum}
  \label{sub:gauss}
    For every natural number $n$ we have:
    \[
      \sum_{i=1}^n i = \frac{n(n+1)}{2}
    \]

    \textit{Proof:}\\
    We set $S(n)\coloneqq 1 + 2 + \ldots + n$ and show $S(n)=\frac{n(n+1)}{2}$ through complete induction.

  \section{More}
    \begin{align}
      \begin{pmatrix}
        n \\
        k
      \end{pmatrix}
      &\coloneqq
      \prod_{j=1}^k \frac{n-j+1}{j} \\
      &= \frac{n(n-1)\cdot\ldots\cdot(n-k+1)}{1\cdot 2\cdot \ldots \cdot k}
    \end{align}
    From the definition we know the following.
    \begin{equation}
      \begin{pmatrix}
        n \\ 0
      \end{pmatrix}
      = 1,\qquad
      \begin{pmatrix}
        n \\ 1
      \end{pmatrix}
      = n
    \end{equation}
    Of course we know a lot from section \ref{sub:gauss}.

  \subsection{Logical...}
    \[
      x < y \quad \iff \quad -x > -y
    \]
\end{document}