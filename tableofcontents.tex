\documentclass[a4paper]{article}
% \documentclass[a4paper,twoside]{article}
\usepackage[utf8]{inputenc}
\usepackage[T1]{fontenc}
\usepackage{amsmath}
\usepackage{mathtools}
\usepackage{standalone}
\usepackage{float}
\usepackage{subcaption}
\usepackage{multirow}

% \usepackage[english]{babel}
% \usepackage[square,sort,comma,numbers]{natbib}
% \usepackage{babelbib}
% \bibliographystyle{babplain}


\title{Thesis: Really Important Stuff}
\author{Max Mustermann}

\begin{document}
  \maketitle

  \input{abstract}

  \tableofcontents
  \cleardoublepage
  \listoffigures

  % \newpage
  \cleardoublepage

  \input{abstract}

  \section{Introduction}
    \begin{figure}[H]
      \center
      \includegraphics[width=0.9\textwidth]{example-audi_r8-pt.png}
      \caption[Audi Car]{This figure shows a cool car!}
      \label{fig:audi}
    \end{figure}

    The figure \ref{fig:audi} shows a cool car which was generated by the computer.

  \section{Background}
    \begin{figure}[H]
      \center
      \includegraphics[width=0.9\textwidth]{brdf_1.pdf}
      \caption{This one is a scheme.}
      \label{fig:brdf}
    \end{figure}

  \section{Methodology}
    \begin{figure}
      \center
      \begin{subfigure}[b]{0.49\textwidth}
        \center
        \includegraphics[width=0.9\textwidth]{brdf_1.pdf}
        \caption{first}
      \end{subfigure}
      \begin{subfigure}[b]{0.49\textwidth}
        \center
        \includegraphics[width=0.9\textwidth]{example-audi_r8-pt.png}
        \caption{second}
      \end{subfigure}

      \begin{subfigure}[b]{0.99\textwidth}
        \center
        \includegraphics[width=0.9\textwidth]{example-audi_r8-pt.png}
        \caption{third}
      \end{subfigure}
      \caption[Wow!]{Two images!}
    \end{figure}

  \section{Implementation}
  \section{Measurement}
  \section{Results}

    \begin{table}[H]
      \center
      \caption{testtable}
      \begin{tabular}{llll}
        \hline
        One & Two & Three & Four \\
        \hline
        \hline
        1 & 2 & 3 & 4 \\
        text & more text & only text & nah... \\
        \hline
      \end{tabular}
    \end{table}

    \begin{table}[H]
      \center
      \caption{testtable}
      \begin{tabular}{lrcl}
        \hline
        One & Two & Three & Four \\
        \hline
        \hline
        1 & 2 & 3 & 4 \\
        text & more text & only text & nah... \\
        \hline
      \end{tabular}
    \end{table}

    \begin{table}[H]
      \center
      \caption{testtable}
      \begin{tabular}{|l|r|cl|}
        \hline
        One & Two & Three & Four \\
        \hline
        \hline
        1 & 2 & 3 & 4 \\
        text & more text & only text & nah... \\
        \hline
      \end{tabular}
    \end{table}

    \begin{table}[H]
      \center
      \begin{tabular}{lll}
        \hline
        one & two & three \\
        \hline
        \hline
        \multirow{2}{*}{$n$} & $p$ & $q$ \\
          & $a$ & $b$ \\
        \hline
      \end{tabular}
    \end{table}

    \begin{table}[H]
      \center
      \begin{tabular}{lll}
        \hline
        one & two & three \\
        \hline
        \hline
        \multirow{2}{*}{$n$} & \multirow{2}{*}{$p$} & $q$ \\
          & & $b$ \\
        \hline
      \end{tabular}
    \end{table}

  \section{Conclusion}

  % \nocite{*}
  % \bibliography{references}

  \appendix
  \section{Extra Results}
  \section{More Background}
\end{document}